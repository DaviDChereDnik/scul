\documentclass[14pt, a4paper]{article}

%%%%%%%%%%%%%%%%%%%%%%%%%%%%%%%%%%%%%%%%%%
% > Packages
%%%%%%%%%%%%%%%%%%%%%%%%%%%%%%%%%%%%%%%%%%
% Set utf-8 encoding
\usepackage[utf8]{inputenc}
% Adding full support of english lang
\usepackage[russian]{babel}
% Editing titles
\usepackage{titlesec}

%%%%%%%%%%%%%%%%%%%%%%%%%%%%%%%%%%%%%%%%%%
% > Settings 
%%%%%%%%%%%%%%%%%%%%%%%%%%%%%%%%%%%%%%%%%%
% Edit section font size and style
\titleformat{\section}
{\huge\bfseries}{\thesection.}{1em}{}
% Command for presentation's slides description
\newcommand{\descr}[1]
  {\par\noindent\rule{0.5\textwidth}{0.4pt} \par {\large #1}}


%%%%%%%%%%%%%%%%%%%%%%%%%%%%%%%%%%%%%%%%%%
% > Document
%%%%%%%%%%%%%%%%%%%%%%%%%%%%%%%%%%%%%%%%%%
% Title info
\title{\Huge\bfseries Темные места \\ \Huge\bfseries "Слова о полку Игореве"}
\author{\huge Чередник Давид \\ \huge Валерий Древлянский \\\\ \Large 9M класс}
\date{}

\begin{document}

\clearpage\maketitle
\begin{center}
  \itshape 16.10.2020
\end{center}
\thispagestyle{empty}

\newpage

\tableofcontents

\newpage

{\Large

\section{Титульный слайд}
Всех приветствую. Сегодня мне бы хотелось поговорить о проблемах и темных местах текста "Слова о полку Игореве".
\descr{
  Стандартный титульный слайд - полное название темы доклада, авторы, класс, школа, дата выступления.}

\section{Вступление}
Впервые "Слово о полку Игореве" было обнаружено только в 18 веке, при том что оригинал был написан предположительно в 12 веке. И с тех пор, как его обнаружили, неутехают споры о так называемых "темных местах" в этом произведении - местах с непонятным или многозначным толкованием или переводом. Но такие места появились неспроста:
\begin{enumerate}
  \item Во первых, большую роли играет такой большой временной разрыв между временем написания и временем обнаружения, ведь за примерно 6 веков в России произошло много событий, повлиявших на культуру страны и на культуру языка в частности.
  \item Во вторых, не менее важную роль сыграл способ записи в 12 веке - в целях экономии бумаги они записывали все сплошным текстом, не деля не то что абзацы, а даже слова.
  \item Третья, но не последняя по важности причина - все тексты, изычаемые сейчас - копии оригинала, а пре переписывании оригинала также допускались ошибки, что добавило еще больше работы лингвистам.
\end{enumerate}
\par Мы предлагаем рассмотреть некоторые из таких мест.
\descr{
  Картинка слова (можно как оригинал, так и копию), а также (можно и не делать) - весь текст этого слайда в кратких фактах}

\section{Не лЪпо ли ны бяшетъ, бра­тіе, начяти…}
\par Уже первая строка произведения - «Не лЪпо ли ны бяшетъ, бра­тіе, начяти…» - вызывает дискуссии. 
Вот толкования этого выражения:
ТОЧНО ЛИ НАДО ДЕЛАТЬ ИМЕННО СПИСКОМ?
\begin{itemize}
  \item В. Л. Жуковский, признанный мастер художественного перевода, интерпретирует эту строку так: «Не прилично ли будет нам, братья, начать…», ставя в конце предложения восклицательный знак. 
  \item Поэт-символист рубежа XIX-XX веков К. Д. Бальмонт считает иначе - он предлагает более привычное для читателя поэтическое звучание: «Нам начать не благо ль, братья…», заканчивая строку вопросительным знаком.
  \item Перевод Н. А. Заболоцкого, часто включаемый в школьные учебники, выглядит так - "Не пора ль нам, братия, начать...", тоже с вопросительным знаком, как у В. Л. Жуковского.
  \item Очень интересный вариант перевода у А. К. Югова, исседователя "Слова" - "А было бы лепо нам, братья, сказать", сохраняя древнерусское слово "лепо", обосную это тем, что оно хорошо понятно сейчас и до сих пор широко используется в различных славянских языках. Этим самым он сохраняет особую поэтику "Слова". В плане того, считать ли это высказывание вопросительным или восклицательным, Югов считает что частица "ли" имеет усилительное, и следовательно восклицательное значение, а значит и все это высказывание восклицательное. Самое интересное, что Пушкин придерживался именно мнения А. К. Югова.
\end{itemize}
\descr{
Краткое описание и написание всех вариантов, а также написание самого высказывания в оригинале. Можно также какую нибудь древнерусскую картинку с братьями}

\section{растЪкашется мыслію по древу}
\par Фраза "растЪкашется мыслію по древу" представляет собой тот случай, когда перевод довольно прямолинеен и однозначен, а толкование непонятно.
\par ЧТО ТУТ БЛЯТЬ ПИСАТЬ?
\begin{itemize}
  \item Академик Лихачев считает, что это книжный образ, а не народно-поэтический, как некоторые могут подумать. Аргументирует он это тем, что такое поэтической манерой древний певец-сказатель Боян обозначал "мысленное дерево". Соответственно, Лихачев переводит это как "растекался мыслью по дереву". Большинство современных лингвистов склоняются, хотя есть и другие версии.
  \item В 19 веке самым популярным толкованием было толкование Е. В. Барсова. Он считает, что во время переписывания этого произведения в 18 веке, была допусчена ошибка - переписчики не смогли различить написанные слитно два слова, из которых одно было под титлом (специальным значком, поставленным сверху слова, который позво­лял в древнерусских текстах сокращать слово до нескольких букв). Неправильное прочтение привело к тому, что они сочли ошибкой написанные рядом два очень похожих слова: «мыслиюмысию» — и оставили из них одно: «мыслию».
  \par Но, по его мнению, тут на самом деле существует два слова, причем одно из них - "мысию" - означает белку. Соответственно, перевод Барсова звучит так - "разлетается мыслью-белкою по древу".
  \item А. К. Югов внес небольшую корректировку в версию Барсова - по его мнению, под мыслью писатель подразумевал мышь. На самом деле, существовал особый вид древесной мыши, обитавший только на Древней Руси. Но у современных лингвистов большие сомнения, можно ли связывать лингвистическое толкование с такими зоологическими объяснениями.
\end{itemize}
\descr{
С этого слайда и так далее - так же как в третьем слайде.}

\section{Троян}
В тексте "Слова о полку Игореве" слово "троян" встречается четыре раза:
\begin{itemize}
  \item "рища в тропу Трояню"
  \item "были веце Трояни"
  \item "на земле Трояню"
  \item "на седьмом веце Трояни"
\end{itemize}
\par ЧТО ТУТ ПИСАТЬ БЛЕАТ?
\begin{itemize}
    \item Н. С. Тихонравов удтверждает, что была совершенна ошибка при переписывании и в оригинале было "Бoян". С. К. Шамбинаго подтверждает его теорию, объясняе это тем, что первые переписчики затруднялись в чтении оригинальной лигатуры "Тр", и в первых версиях переписывания для Екатерины II его записывали как "Зоянь". Но, в западнорусской графике буквы "б" и "з" и лигатура "Тр" пишутся одинаково. Из этого он сделал вывод, что вместо "Троян" в оригинале был "Боян". Этого мнения также придерживается Югов.
    \item Несмотря на такую логичность теории Н. С. Тихонравова и С. К. Шамбинаго, все больше и больше современных лингвистов придерживаются мнения, что в оригинале было все таки "Троян". Академик Лихачев говорит, что под "Трояном" подразумевается языческий бог. Тогда можно довольно легко истолковать те 4 выражения:
    \begin{itemize}
      \item "рища в тропу Трояню" - "носясь по божественным путям"
      \item "были веце Трояни" - "были века язычества"
      \item "на земле Трояню" - "на Русскую землю"
      \item "на седьмом веце Трояни" - "на последнем веке язычества"
    \end{itemize}
    У вас может возникнуть вопрос - почему, если Троян означает языческого бога, то выражение "на земле Трояню" расшиврововается не как "на божественной земле", а как "на Русскую землю"? Лихачев объяснеят это тем, что дальше, в "Слове", русский народ называется Даждьбожим внуком.
    \par Это теория еще больше подтверждается тем, что в "Слове" также упоминаются другие языческие боги.
\end{itemize}
У вас также может возникнуть вопрос, почему если все действия "Слова" происходит после крещения Руси, автор использует языческих богов. Ответ довольно прост - он использует их не как богов, а как поэтические символы (таким же образом в поэзии 18 века использовались божества античности).

\section{ДивЪ}
Вот мы и подобрались к самому интересному и сложному месту в "Слове о полку Игореве" - слово "дивЪ". Вот фрагмент текста "Слова", где используется это слово: {\itshape "Тогда въступи Игорь Князь въ златъ стремень, и Солнце ему тъмою путь заступаша; нощь стонуши ему грозою птичь убуди; свистъ зверинъ въ стазби; {\bfseries дивъ} кличетъ връху древа, велитъ послушати земли незнаемъ, влъзе, и по морію, и по Сулію, и Сурожу, и Корсуню, и тебе Тьмутораканьскый блъванъ"}
\par AAAAAAAAAAAAAA
\begin{itemize}
  \item Д. С. Лихачев предлагает такой перевод этого фрагмента: {\itshape "Тогда вступил Игорь-князь в золотое стре­мя и поехал по чистому полю. Солнце ему тьмою путь засту­пало; ночь стонами грозы птиц пробудила; свист звериный встал, взбился; {\bfseries див} — кличет на вершине дерева, велит прислушаться — земле незнаемой, Волге, и Поморью, и Посулью, и Сурожу, и Корсуню, и тебе, Тмутороканский идол!}. Дополнительно к этому переводу он прилагает пояснение - {\itshape "Слово “див” не получило обще­признанного объяснения. Большинство исследователей счита­ет “дива” мифическим существом (чем-то вроде лешего или вещей птицы). В “Слове о полку Игореве” “див” предупреждает враждебные Руси страны, это божество восточных на­родов, сочувствующее им, а не Руси.}
  \item Противником теории Лихачева выступил Югов - он считает, что нужно {\itshape "историко-лингвистическое понимание, вопреки застоявшемуся с мусин-пушкинских времен “орнитологическому” (птицеведческому) и мифологическому толкованию. … Голосистая же, можно сказать, птица, этот филин, если слыхать было его за тысячи верст! И никто не задумывается над тем, что гениальный поэт, автор “Слова”, никак не мог написать столь нелепый образ, да еще прибегнув к нему в грозно-величественных строфах, где русское воинство Игоря вступает во вражеское “поле Поло­вецкое”!"}
  \par Он выдвигает теория, что была допущена очередная ошибка пре переписывание, и в оригинале было на "дивЪ" а "дивЬ". Таким образом, это слово является собирательным образом таких понятий, как "дикарь" и "дикие" (производное "дивый" - дикий, то есть дивь -> дикие -> половцы). Соответственно этой теории, в оригинале это означало "половецкие орды".
  \item Производной от теории Югова является теория Павлова-Бицина - он считает, что "дивЪ", который кличет с вершины дерева - дозорник или караульщик.
  \item Самая оригинальная трактовка, как я считаю, пренадлежит И. А. Новикову. Вот что он говорит - {\itshape "Див русским {\bfseries не враждебен}. Его клич “земле незнаемой” скорее подобен кличу Святослава Игоре­вича: “Иду на вы!” Заслышав клич Дива, половцы, подобно распуганным лебедям, кинулись назад — к Дону великому"}. 
    \par Основоваясь на этом, он выдвигает свое предположение, что див не сидит на дереве, а на древке знамени Игоря, то есть является {\bfseries изображением}. Но что же это значит. Новиков считает, что Диво - нечто чудесное, друг и союзник русских. Таким образом он решает, что это может быть изображение {\itshape крылатого архистратига небесных сил} - {\bfseries Архангела Михаила}, потому что именно так его изображали от конца 12 до начала 13 века.
\end{itemize}
\descr{F тебе, делая слайд или слайды под все это :)}

\section{Вывод}
Мы рассмотрели самые интересные и простые для понимания темные места "Слова", но это не является даже сотой частью. На примере слова можно рассмотреть, как меняется язык с течением времени, как люди из одного поколения усложняют работу будущим поколениям и откуда лингвисты получают деньги (ЧТО ТУТ?)



}
\end{document}
